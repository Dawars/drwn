% DARWIN DOCUMENTATION CHEATSHEET
% Stephen Gould <stephen.gould@anu.edu.au>

\documentclass[landscape,8pt]{article}
\usepackage{amssymb,amsmath,amsfonts,latexsym}

% page formatting
\usepackage[landscape,letterpaper,left=1.0cm,top=0.5cm,right=1.0cm,bottom=1.0cm,nohead,nofoot]{geometry}
\usepackage[compact,small]{titlesec}
\titlespacing{\section}{0pt}{*2}{*0}
\titlespacing{\subsection}{0pt}{*0}{*0}
\pagestyle{empty}

\setlength{\parindent}{0pt}
\setlength{\parskip}{5pt}

\usepackage[pagebackref=false,breaklinks=true,colorlinks,bookmarks=false]{hyperref}

\usepackage{multicol}
\setlength{\columnseprule}{0.4pt}

% rules
\newcommand{\thickhline}{\noalign{\hrule height 0.8pt}}

\begin{document}

% heading -------------------------------------------------------------------------------

\begin{center}
  \Large
  {\sc Darwin}: A Framework for Machine Learning and Computer Vision Research (v1.0)
\end{center}

\small
\begin{multicols}{3}

% overview ------------------------------------------------------------------------------

\section*{Overview}

This note provides a quick reference---via examples---for the main
features of the {\sc Darwin} framework for machine learning and
computer vision research. Complete documentation is available at
\url{http://drwn.anu.edu.au}.

% makefiles -----------------------------------------------------------------------------

\section*{Building Applications}

The following example shows a very simple {\sc Darwin} application
that prints ``hello world'' and exits.

\begin{tabular}{c}
\thickhline
{\sc code.cpp}\\
\hline
\begin{minipage}[h]{0.95\columnwidth}
\smallskip
\begin{verbatim}
#include <cstdlib>
#include <cstdio>

#include "drwnBase.h"

int main(int argc, char *argv[]) {
    DRWN_LOG_MESSAGE("hello world");
    return 0;
}
\end{verbatim}
\smallskip
\end{minipage}\\
\thickhline
\end{tabular}

The code can be compiled with the following \texttt{Makefile}.

\begin{tabular}{c}
\thickhline
{\sc Example Makefile}\\
\hline
\begin{minipage}[h]{0.95\columnwidth}
\smallskip
\begin{verbatim}
INCS = -I${DARWIN}/include -I${DARWIN}/external
LIBS = -L${DARWIN}/bin -ldrwnBase -lm -lpthread

main:
    g++ -g -o app code.cpp ${INCS} ${LIBS}
\end{verbatim}
\smallskip
\end{minipage}\\
\thickhline
\end{tabular}
\smallskip

% command line processing ---------------------------------------------------------

\section*{Command Line Processing}

Standard options for most {\sc Darwin} applications are:
\begin{verbatim}
 -help            :: display application usage
 -config <xml>    :: configure Darwin from XML
 -set <m> <n> <v> :: set value of <m>::<n> to <v>
 -profile         :: profile code
 -quiet           :: only show warnings and errors
 -verbose         :: show verbose messages
 -debug           :: show debug messages
 -log <filename>  :: log filename
 -threads <max>   :: set maximum number of threads
\end{verbatim}

Command line options are processed from left to right. If there are
multiple conflicting options, the rightmost one will be taken, e.g.,
\texttt{-threads 4 -threads 0} will result no multi-threading.
Many {\sc Darwin} features can be configured using the
\texttt{-config} or \texttt{-set} command line options. The standard
options provide shortcuts for these. E.g., \texttt{-verbose} is
equivalent to \texttt{-set drwnLogger logLevel VERBOSE}.

Include a \texttt{DRWN\_BEGIN\_CMDLINE\_PROCESSING} block to
automatically handle these options in your applications.

% libraries -----------------------------------------------------------------------------

\section*{drwnBase Library}

The {\bf drwnBase} library provides a number of core utility classes
such as command line processing, code profiling, and message
logging. All applications that use the {\sc Darwin} framework must
include \texttt{drwnBase.h} and link against this library.

\texttt{drwnCodeProfiler} provides simple code profiling that can be
controlled via the \texttt{-profile} command line option.

\begin{tabular}{c}
\thickhline
{\sc Code Profiling}\\
\hline
\begin{minipage}[h]{0.95\columnwidth}
\smallskip
\begin{verbatim}
double veryLongCalculation() {
    DRWN_FCN_TIC; 
    result = 0.0; // do calculation
    DRWN_FCN_TOC;
    return result;
}
\end{verbatim}
\smallskip
\end{minipage}\\
\thickhline
\end{tabular}
\smallskip

Messages, warnings and errors are managed via the \texttt{drwnLogger}
class. The \texttt{DRWN\_LOG\_*} family of macros will automatically
write log messages to a file (if specified by the \texttt{-log}
command line option) and display them on the console. You can set the
verbosity level to control which messages get displayed.

\begin{tabular}{c}
\thickhline
{\sc Message Logging}\\
\hline
\begin{minipage}[h]{0.95\columnwidth}
\smallskip
\begin{verbatim}
  DRWN_LOG_MESSAGE("start loop");
  for (int i = 0; i < 10; i++) {
    DRWN_LOG_VERBOSE("iteration " << i);
  }

  DRWN_ASSERT_MSG(a != b, "assert message");
\end{verbatim}
\smallskip
\end{minipage}\\
\thickhline
\end{tabular}
\smallskip

{\sc Darwin} makes extensive use of XML formatting for serialization
(saving) and de-serialization (loading) of objects. All objects
derived from \texttt{drwnWriteable} implement methods for saving state
to, and loading state from, an XML object.

A number of helper functions are provided in
\texttt{drwnXMLUtils}. The following code snippet shows an example:

\begin{tabular}{c}
\thickhline
{\sc XML Example}\\
\hline
\begin{minipage}[h]{0.95\columnwidth}
\smallskip
\begin{verbatim}
class MyObject {
public:
    VectorXd x;

    MyObject() { x = VectorXd::Random(5); }

    void save(drwnXMLNode& xml) const { 
        drwnXMLUtils::serialize(xml, x); }
    void load(drwnXMLNode& xml) {
        drwnXMLUtils::deserialize(xml, x); }
};

int main() {
    // create a vector of MyObject
    vector<MyObject> v(10);
    // save container to an XML file
    drwnXMLUtils::write("eg.xml", "vec", "obj", v);

    return 0;
}
\end{verbatim}
\smallskip
\end{minipage}\\
\thickhline
\end{tabular}
\smallskip

\section*{drwnIO Library}

The {\bf drwnIO} library provides input/output functionality such as
data storage in uncompressed or compressed format (provided by
\texttt{zlib}). Applications should include the \texttt{drwnIO.h}
header file.

\section*{drwnML Library}

The {\bf drwnML} library provides basic machine learning capability
for optimization, classification, regression and modeling probability
distributions. Applications should include the \texttt{drwnML.h}
header file.

\section*{drwnPGM Library}

The {\bf drwnPGM} library provides infrastructure for inference and
learning in probabilistic graphical models. Applications should
include the \texttt{drwnPGM.h} header file.

\section*{drwnVision Library}

The {\bf drwnVision} library provided high-level computer vision
routines. It requires OpenCV (\url{http://opencv.willowgarage.com}).
Applications should include the \texttt{drwnVision.h} header
file. \emph{This library is optional.}

\end{multicols}
\end{document}
